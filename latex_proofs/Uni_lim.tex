\documentclass{article}
\usepackage{amsmath}
\usepackage{amsfonts}
\title {Unicité de la limite d'une suite numérique}
\author{Gabriel Vandevoorde}


\begin{document}

\maketitle
%Ceci est un commentaire
Soit $ (u_{n})_{n \in \mathbb{N}} \in \mathbb{R}^{\mathbb{N}}$.
On suppose qu'on dispose de $ l \in \mathbb{R} $ tel que $ u_{n} \xrightarrow[n \rightarrow \infty]{} l$

Ce $l \in \mathbb{R}$ est unique, et appelé limite de la suite $(u_{n})_{n \in \mathbb{R}}$ 
\newline On raisonne par l'absurde. Supposons que l'on dispose de $l_{1}, l_{2} \in \mathbb{R} $ tels que 

$ u_{n} \xrightarrow[n \rightarrow \infty]{} l_{1}$ et $ u_{n} \xrightarrow[n \rightarrow \infty]{} l_{2}$ 

Alors d'après la définition de la convergence d'une suite, en posant $\epsilon= \frac{|l_{1} - l_{2}|}{4} > 0$
On dispose de $N_{1},N_{2} \in \mathbb{N}$ tels que, pour $n >= N_{1}$ on ait $u_{n} \in [l_{1}-\epsilon , l_{1} + \epsilon  ] $
et pour $n >= N_{2}$ $u_{n} \in [l_{2}-\epsilon , l_{2} + \epsilon  ] $

Sans perte de généralités, on peut supposer que $l_{1} < l_{2}$, on a alors
en posant $N = max(N_{1},N_{2}) $ il vient que $\forall n >= N$, $u_{n} \in [l_{1}-\epsilon , l_{1} + \epsilon  ] \cap [l_{2}-\epsilon , l_{2} + \epsilon  ]$
Or $[l_{1}-\epsilon , l_{1} + \epsilon  ] \cap [l_{2}-\epsilon , l_{2} + \epsilon  ] = \emptyset $

Ceci est absurde, achevant donc la démonstration.

 
 
 

\end{document}
